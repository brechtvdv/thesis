%  Overzichtsbladzijde met samenvatting

\newpage

{
\setlength{\baselineskip}{14pt}
\setlength{\parindent}{0pt}
\setlength{\parskip}{8pt}

\begin{center}

\noindent \textbf{\huge
Optimalisatie van client-side\\[8pt]
intermodale routeplanning
}

door 

Brecht Van de Vyvere

Scriptie ingediend tot het behalen van de academische graad van\\
Master of Science in de industri\"ele wetenschappen: informatica

Promotor: prof.dr.ir.~Rik~Van de Walle, prof.dr.ir.~Erik~Mannens\\
Scriptiebegeleider: dr.~ir.~Ruben~Verborgh, ~Pieter~Colpaert

Vakgroep Elektronica en Informatiesystemen, Vakgroep Industri\"ele Technologie en Constructie\\
Voorzitter: Prof.~Dr.~Ir.~Rik~Van de Walle\\
Faculteit Ingenieurswetenschappen en Architectuur\\
Academiejaar 2015--2016


\end{center}

\section*{Samenvatting}

% TODO: samenvatting

Routeplanning is niet meer weg te denken uit ons dagelijks leven. Is het nu voor de trein naar het werk te nemen of het vliegtuig naar je vakantiebestemming, de mogelijkheden zijn onbeperkt. Sinds enkele jaren wordt transportdata gepubliceerd volgens General Transit Feed Specification (GTFS). Dankzij deze uniforme structuur kunnen er algoritmes bedacht worden om data te combineren en intermodaliteit toe te laten. Bestaande oplossingen maken gebruik van een webservice die zoveel mogelijk vragen van de gebruiker probeert te beantwoorden, maar deze manier is moeilijk uitbreidbaar. \textit{Linked Connections} biedt hier een antwoord op door routeplanning mogelijk te maken op basis van gepubliceerde data. Doordat de server enkel verantwoordelijk is voor het publiceren van connecties is deze makkelijk uitbreidbaar via hypermedia. De cli\"ent berekent zelf welke data nodig is om een om een bepaalde route te kunnen plannen rekening houdend met de eisen van de gebruiker. De huidige implementatie van Linked Connections laat enkel filtering in de tijd toe waardoor de snelheid van het algoritme aan banden ligt. Deze masterproef introduceert een optimalisatie voor routeplanning met Linked Connections.

\section*{Trefwoorden}

% TODO: trefwoorden

Linked Connections, routeplanning, optimalisatie, GTFS

}

\newpage % strikt noodzakelijk om een header op deze blz. te vermijden
