%  Overzichtsbladzijde met samenvatting

\newpage

{
\setlength{\baselineskip}{14pt}
\setlength{\parindent}{0pt}
\setlength{\parskip}{8pt}

\begin{center}

\noindent \textbf{\huge
Optimalisatie van client-side\\[8pt]
intermodale routeplanning
}

door 

Brecht Van de Vyvere

Scriptie ingediend tot het behalen van de academische graad van\\
Master of Science in de industri\"ele wetenschappen: informatica

Promotor: Prof.~Erik~Mannens, Prof.~Rik~Van de Walle\\
Scriptiebegeleider: Dr.~Ir.~Ruben~Verborgh, Ing.~Pieter~Colpaert

Vakgroep Elektronica en Informatiesystemen, Vakgroep Industri\"ele Technologie en Constructie\\
Voorzitter: Prof.~Dr.~Ir.~Rik~Van de Walle\\
Faculteit Ingenieurswetenschappen en Architectuur\\
Academiejaar 2015--2016


\end{center}

\section*{Samenvatting}

% TODO: samenvatting

Routeplanning is niet meer weg te denken uit ons dagelijks leven. Is het nu voor de trein naar het werk te nemen of het vliegtuig naar je vakantiebestemming, de mogelijkheden zijn onbeperkt. In deze thesis wordt onderzocht hoe routeplanning uitbreidbaar, maar ook snel gemaakt kan worden. Sinds enkele jaren wordt transportdata gepubliceerd volgens General Transit Feed Specification (GTFS). Dankzij deze uniforme structuur kunnen er algoritmes bedacht worden om data te combineren en intermodaliteit toe te laten. Bestaande oplossingen maken gebruik van een webservice architectuur, maar dit maakt het moeilijk om te voldoen aan de eisen van de gebruiker. Linked Connections is een manier om routeplanning mogelijk te maken door het publiceren van data. Doordat de server enkel verantwoordelijk is voor het publiceren van deze connecties is deze makkelijk uitbreidbaar via hypermedia. De client zijn zelf intelligent om te beslissen welke data nodig is om een om een bepaalde route te berekenen volgens de eisen van de gebruiker. De huidige implementatie laat enkel filtering in de tijd toe waardoor de client enorm veel data moet verwerken. In deze thesis wordt een optimalisatie ge�ntroduceerd door het toepassen van geolocatie filtering.

\section*{Trefwoorden}

% TODO: trefwoorden

Linked Connections, routeplanning, optimalisatie, GTFS

}

\newpage % strikt noodzakelijk om een header op deze blz. te vermijden
